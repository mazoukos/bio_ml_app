\documentclass[a4paper,12pt]{article}
\usepackage[utf8]{inputenc}
\usepackage[greek,english]{babel}
\usepackage{graphicx}
\usepackage{geometry}
\usepackage{hyperref}
\usepackage{amsmath}
\usepackage{listings}
\usepackage{xcolor}

\geometry{margin=2.5cm}

% Ρυθμίσεις για κώδικα
\lstset{
  basicstyle=\ttfamily\footnotesize,
  keywordstyle=\color{blue},
  commentstyle=\color{green!60!black},
  stringstyle=\color{red},
  breaklines=true,
  frame=single,
  language=Python
}

\title{ 
  \textbf{Διαδραστική Εφαρμογή Ανάλυσης Μοριακής Βιολογίας με Machine Learning} \\
  \large Εαρινό Εξάμηνο 2024-2025
}
\author{
  Μαρία Παπαδοπούλου, Γιώργος Νικολάου, Άννα Σταυροπούλου \\
  Τμήμα Μοριακής Βιολογίας και Γενετικής \\
  Πανεπιστήμιο ...
}
\date{\today}

\begin{document}

\maketitle

\selectlanguage{greek}

\begin{abstract}
Στην παρούσα εργασία παρουσιάζεται η ανάπτυξη μιας διαδραστικής εφαρμογής ανάλυσης δεδομένων μοριακής βιολογίας με τη χρήση μεθόδων μηχανικής μάθησης, υλοποιημένη με το framework Streamlit. Η εφαρμογή επιτρέπει στο χρήστη να φορτώνει δεδομένα, να εκτελεί προεπεξεργασία, να εκπαιδεύει και να αξιολογεί μοντέλα, καθώς και να βλέπει οπτικοποιήσεις των αποτελεσμάτων. Επιπλέον, η εφαρμογή είναι dockerized, επιτυγχάνοντας φορητότητα και εύκολη εκτέλεση σε διάφορα περιβάλλοντα. Στην εργασία παρουσιάζονται η σχεδίαση, η υλοποίηση, οι οπτικοποιήσεις και η διαδικασία dockerization.
\end{abstract}

\section{Εισαγωγή}
Η ανάλυση δεδομένων μοριακής βιολογίας αποτελεί ένα κρίσιμο πεδίο έρευνας, όπου οι μέθοδοι μηχανικής μάθησης προσφέρουν ισχυρά εργαλεία για την εξαγωγή γνώσης και την πρόβλεψη χαρακτηριστικών από βιολογικά δεδομένα. Η παρούσα εργασία στοχεύει στην ανάπτυξη μιας φιλικής προς τον χρήστη εφαρμογής που συνδυάζει ανάλυση δεδομένων και μοντέλα μηχανικής μάθησης με διαδραστικό τρόπο, ώστε να υποστηρίξει ερευνητές και φοιτητές στον τομέα.

\section{Σχεδιασμός της Υλοποίησης}

\subsection{Αρχιτεκτονική της Εφαρμογής}
Η εφαρμογή βασίζεται στην Python και το Streamlit για τη δημιουργία του frontend και του backend σε ένα ενιαίο περιβάλλον. Η δομή της περιλαμβάνει τρία βασικά tabs: ανάλυση, οπτικοποιήσεις και πληροφορίες ομάδας. Ο χρήστης μπορεί να φορτώσει δεδομένα ή να χρησιμοποιήσει ένα ενσωματωμένο παράδειγμα, να επιλέξει αλγόριθμο μηχανικής μάθησης και να ρυθμίσει παραμέτρους.

\subsection{Λειτουργικότητα}
Η εφαρμογή υλοποιεί τις εξής διεργασίες:

\begin{itemize}
    \item Φόρτωση δεδομένων από αρχείο CSV ή χρήση έτοιμου dataset (Breast Cancer).
    \item Προεπεξεργασία δεδομένων με κανονικοποίηση.
    \item Εκπαίδευση μοντέλου (Random Forest, SVM, KNN) με διαχωρισμό σε train/test.
    \item Αξιολόγηση μοντέλου με classification report και confusion matrix.
    \item Οπτικοποιήσεις με PCA και heatmap.
\end{itemize}

\section{UML Διαγράμματα}

\subsection{Use Case Diagram}
\begin{figure}[h]
  \centering
  \includegraphics[width=0.8\textwidth]{use_case_diagram.png}
  \caption{Use Case Diagram της εφαρμογής}
\end{figure}

\subsection{Class Diagram}
\begin{figure}[h]
  \centering
  \includegraphics[width=0.8\textwidth]{class_diagram.png}
  \caption{Class Diagram της αρχιτεκτονικής}
\end{figure}

\section{Ανάλυση Υλοποίησης}

Παρακάτω παρουσιάζονται αποσπάσματα του κώδικα και περιγραφή των σημαντικότερων μερών:

\subsection{Φόρτωση και Προεπεξεργασία Δεδομένων}
\begin{lstlisting}
import pandas as pd
from sklearn.preprocessing import StandardScaler

df = pd.read_csv('data.csv')
X = df.iloc[:, :-1]
y = df.iloc[:, -1]

scaler = StandardScaler()
X_scaled = scaler.fit_transform(X)
\end{lstlisting}

\subsection{Εκπαίδευση Μοντέλου}
\begin{lstlisting}
from sklearn.ensemble import RandomForestClassifier
from sklearn.model_selection import train_test_split

X_train, X_test, y_train, y_test = train_test_split(
    X_scaled, y, test_size=0.2, random_state=42)

clf = RandomForestClassifier(random_state=42)
clf.fit(X_train, y_train)
y_pred = clf.predict(X_test)
\end{lstlisting}

\subsection{Αξιολόγηση και Οπτικοποίηση}
\begin{lstlisting}
from sklearn.metrics import classification_report, confusion_matrix
import matplotlib.pyplot as plt
import seaborn as sns
from sklearn.decomposition import PCA

print(classification_report(y_test, y_pred))

pca = PCA(n_components=2)
X_pca = pca.fit_transform(X_scaled)

sns.scatterplot(x=X_pca[:,0], y=X_pca[:,1], hue=y)
plt.show()

cm = confusion_matrix(y_test, y_pred)
sns.heatmap(cm, annot=True, fmt='d')
plt.show()
\end{lstlisting}

\section{Οπτικοποιήσεις Αποτελεσμάτων}

\begin{figure}[h]
  \centering
  \includegraphics[width=0.7\textwidth]{pca_scatter.png}
  \caption{Οπτικοποίηση PCA για το dataset}
\end{figure}

\begin{figure}[h]
  \centering
  \includegraphics[width=0.7\textwidth]{confusion_matrix.png}
  \caption{Confusion matrix από το μοντέλο}
\end{figure}

\section{Dockerization της Εφαρμογής}

Η εφαρμογή τρέχει μέσα σε Docker container για να εξασφαλίσει φορητότητα και ανεξαρτησία από το περιβάλλον.  

\subsection{Dockerfile}

\begin{lstlisting}[language=bash]
FROM python:3.10

WORKDIR /app
COPY . /app

RUN pip install --no-cache-dir -r requirements.txt

EXPOSE 8501

CMD ["streamlit", "run", "app.py", "--server.port=8501", "--server.address=0.0.0.0"]
\end{lstlisting}

\subsection{Οδηγίες Εκτέλεσης Docker}

\begin{itemize}
  \item Δημιουργία image: \texttt{docker build -t bioml-app .}
  \item Εκτέλεση container: \texttt{docker run -p 8501:8501 bioml-app}
\end{itemize}

\section{Συμπεράσματα}

Η ανάπτυξη της εφαρμογής κατέδειξε την αποτελεσματικότητα του Streamlit ως εργαλείου για γρήγορη και φιλική προς το χρήστη παρουσίαση εργαλείων μηχανικής μάθησης. Η χρήση Docker διασφαλίζει ότι η εφαρμογή μπορεί να τρέξει οπουδήποτε χωρίς πρόβλημα ρυθμίσεων. Μελλοντικές βελτιώσεις μπορούν να περιλαμβάνουν υποστήριξη περισσότερων αλγορίθμων και μεγαλύτερη παραμετροποίηση.

\vspace{1cm}

\noindent
\textbf{Ομάδα:}  
Ιωάννης Νταιλάκης – ML Υλοποίηση  \\
Ιωάννης Μάζης – Streamlit και Docker  \\ 
Στάυρος Ρουμελιώτης – Report και UML Διαγράμματα

\end{document}
